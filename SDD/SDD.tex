\documentclass{article}

\title{Web-based Address Book - Software Design Document}
\date{2016-11-07}
\author{Allen Burgett, Anthony Tyler, Benjamin Altman}

\begin{document}

\pagenumbering{gobble}
\maketitle
\pagebreak

\pagenumbering{arabic}

\section{Summary}
This is a web application that allows the user to store and modify their address book contacts.

\section{Structure}
The program is split into three main parts: An HTML layout found in index.html, a CSS layout found in styles.css, and a JavaScript controller found in javascript.js.

\subsection{index.html}
This file defines the general makeup of information to be displayed on the site. It's purpose is to act as a location template for where the css and javascipt will place, edit, and style the data on the page. Using bootstrap tools, the css is integrated into this file. 

\subsection{javascript.js}
This file contains code that handles data population and editing. It also contains code for user interface options, like buttons and selection. 

\subsubsection{removeContacts}
Removes all contacts.

\subsubsection{getCurrentData}
Pulls contact data from the current list of contacts and returns them as a JavaScript Array Object,

\subsubsection{makeNewContact}
Pulls user input and builds an new contact object. 

\subsubsection{contactClickEvent}
Listens for click on a contact, then displays the contact information by calling endEditDisplay() and setTextDisplay().

\subsubsection{setTextDisplay}
Posts contact information on page for non-editable display.

\subsubsection{setEditDisplay}
Posts contact information on page for editable display

\subsubsection{getEditDisplay}
Pulls data from the non-editable display, to post to the editable display.

\subsubsection{saveButtonHandler}
On save button press, switchs back to display mode for text, and either saves the current displayed information to the current contact, or saves it as a new contact if there is no selected contact.

\subsubsection{newContactHandler}
Dumps any current information on the page by calling in emptyDisplay(). Then, switches to the editable display by calling startEditDisplay().



\end{document}